\documentclass[conference]{IEEEtran}
\IEEEoverridecommandlockouts

\usepackage{cite}
\usepackage{amsmath,amssymb,amsfonts}
\usepackage{algorithmic}
\usepackage{graphicx}
\usepackage{textcomp}
\usepackage{xcolor}
\usepackage{csvsimple}
\usepackage{rotating}
\def\BibTeX{{\rm B\kern-.05em{\sc i\kern-.025em b}\kern-.08em
		T\kern-.1667em\lower.7ex\hbox{E}\kern-.125emX}}

\begin{document}
	
	\title{Efficient Generation of Mandelbrot Set using Message Passing Interface}
	
	\author{
		\IEEEauthorblockN{28993373 \\ Samarasekara Vitharana Gamage, Bhanuka Manesha
		\IEEEauthorblockA{\textit{School of Information Technology} \\
			\textit{Monash University}\\
			Malaysia\\
			bsam0002@student.monash.edu}
		}
	}
	\maketitle
	
	\begin{abstract}
		This document is a model and instructions for \LaTeX.
		This and the IEEEtran.cls file define the components of your paper [title, text, heads, etc.]. *CRITICAL: Do Not Use Symbols, Special Characters, Footnotes, 
		or Math in Paper Title or Abstract.
	\end{abstract}
	
	\begin{IEEEkeywords}
		component, formatting, style, styling, insert
	\end{IEEEkeywords}
	
	\section{Introduction}

	The Mandelbrot set, named after Benoit Mandelbrot is the set of points c in the complex plane which produces a bounded sequence with the application of equation \ref{MandelbrotEq} repeatedly to the point $z=0$ \cite{Mandelbrotset}.
	
	\begin{equation}
		f(z)=z^{2} + c \label{MandelbrotEq}
	\end{equation}
	
	So in order to generate the Mandelbrot set for a $m \times n$ image $I$, we need to execute equation \ref{MandelbrotEq} on each pixel. So we derive the following equation.
	
	\begin{equation}
				M_{i,j} = z_{i,j}^2 + c_{i,j}  \label{MandelPixel}
	\end{equation}
	
	\begin{center}
	where $i = 1...m, j = 1...n$
	\end{center}


	
	
	
	
	
	
	
	
	\section{Preliminary Analysis}
	
			\subsection{Parallelizability of the Generation of Mandelbrot Set}
			
			In order to ensure that the process of generating the Mandelbrot Set is data parallalizable, we can use Bernstein’s conditions \cite{bernstein1996analysis}. Bernstein's condition is a simple verification for deciding if operations and statements can work simultaneously without changing the program output and allow for data parallelism \cite{Feautrier2011}. According to Bernstein, if two processes satisfy the following equations, then they are parallalizable.
			
			\begin{equation}
				I_1 \cap O_2 = \emptyset\label{anti independency}
			\end{equation}
			\begin{equation}
				I_2 \cap O_1 = \emptyset \label{flow independency}
			\end{equation}
			\begin{equation}
				O_1 \cap O_2 = \emptyset \label{output independency}
			\end{equation}
			
			$I_0$ and $I_1$ represents the inputs for first and second process while  $O_0$ and $O_1$ represents the outputs from first and second process. Equation \ref{anti independency} also known as anti in-dependency, states that the input of the first process should not have any dependency with the output of the second process, while equation \ref{flow independency}, also known as flow in-dependency, states that the input of the second process should not have any dependency with the output of the first process. And finally equation \ref{output independency} also known as output in-dependency, states that both the outputs of the two processes should not be equal.
			
			Using equation \ref{MandelPixel} we can derive the input and output equations for any two neighboring pixels which will be computed by two processes $P_{1}$ and $P_{2}$.
			
			\begin{equation}
			I_1 = z_{i,j}^2 + c_{i,j} \label{I1}
			\end{equation}
			\begin{equation}
			O_1 = I_{i,j} \label{O1}
			\end{equation}
			\begin{equation}
			I_2= z_{i,j+1}^2 + c_{i,j+1} \label{I2}
			\end{equation}
			\begin{equation}
			O_2= I_{i,j+1} \label{O2}
			\end{equation}
			
			Applying Bernstein conditions for equations \ref{I1}, \ref{I2}, \ref{O1} and \ref{O2}, we get,
			
			\begin{equation}
			z_{i,j}^2 + c_{i,j}  \quad  \cap \quad I_{i,j+1} = \emptyset
			\end{equation}
			\begin{equation}
			z_{i,j+1}^2 + c_{i,j+1} \quad  \cap \quad I_{i,j} = \emptyset 
			\end{equation}
			\begin{equation}
			I_{i,j}  \quad  \cap \quad  I_{i,j+1} = \emptyset 
			\end{equation}
			
			Since all the above conditions are satisfied, we can state that the generation of Mandelbrot Set satisfies the Bernstein conditions, thereby the data can be parallelized  and computed concurrently. 
			
			Next we calculate the theoretical speed up to determine whether there is a benefit of running the code in parallel.
			
			
			\subsection{Theoretical Speed Up of the Generation of Mandelbrot Set}
			
			To determine the theoretical speed up of the Mandelbrot Set Generation, we first determine the pararalizable portion of the serial implementation of the code. This can be achieved by calculating the total time for generating the Mandelbrot set ($t_p$) and the total time to execute the whole serial code ($t_t$). Then using equation \ref{rp} we can get the $r_p$ value.
			
			\begin{equation}
				r_p = \frac{\text{time taken for parallelizable portion of the code}}{\text{total time for serial execution}} \label{rp}
			\end{equation}
			
			Then after calculating the $r_p$ value we can use Amdahl's Law () to calculate the theoretical speed up value to generate the Mandelbrot set. Amdah stated that if $p$ is the number of processes, $r_s$ is the time spent on the serial part of the code and $r_p$ is the amount of time spent on the part of the program that can be done in parallel, () then the speedup can be stated as 
			
			\begin{equation}
				S_p = \frac{1}{r_s + \frac{r_p}{p}}
			\end{equation}
			
			So using Amdah's law, we are able to generate the following chart with the theoretical speed up values.
			
			\begin{table}[!h]\caption{Number of processes vs Speed up factor}
				\begin{center}
					\begin{tabular}{|c|l|} 
						\hline
						\textbf{Number of Processes} & \textbf{Speed Up Factor} \\ \hline
						4                  & 3.99968                     \\
						8                   & 7.9985                       \\
						16                 & 15.99378                       \\
						32                & 31.9743                       \\
						64                  & 63.89566                       \\
						 $\infty$      & 38580.2469                       \\ \hline
					\end{tabular}
				\end{center}
			\end{table}
			
			So having calculated the theoretical speed up, next we look into the design of the partition scheme for parallelizing data.
			
			
	\section{Design of Partition Schemes}
	Before you begin to format your paper, first write and save the content as a 
	separate text file. Complete all content and organizational editing before 
	formatting. Please note sections  below for more information on 
	proofreading, spelling and grammar.
	
	Keep your text and graphic files separate until after the text has been 
	formatted and styled. Do not number text heads---{\LaTeX} will do that 
	for you.
	
			\subsection{Naive Row Segmentation-based Partition Scheme }\label{Naive}
			Define abbreviations and acronyms the first time they are used in the text, 
			even after they have been defined in the abstract. Abbreviations such as 
			IEEE, SI, MKS, CGS, ac, dc, and rms do not have to be defined. Do not use 
			abbreviations in the title or heads unless they are unavoidable.
			
			\subsection{First Come First Serve Row Segmentation-based Partition Scheme}\label{FCFS}
			Define abbreviations and acronyms the first time they are used in the text, 
			even after they have been defined in the abstract. Abbreviations such as 
			IEEE, SI, MKS, CGS, ac, dc, and rms do not have to be defined. Do not use 
			abbreviations in the title or heads unless they are unavoidable.
			
			
			\subsection{Improving \ref{FCFS} with MPI PACK }\label{FCFS_PACK}
			Define abbreviations and acronyms the first time they are used in the text, 
			even after they have been defined in the abstract. Abbreviations such as 
			IEEE, SI, MKS, CGS, ac, dc, and rms do not have to be defined. Do not use 
			abbreviations in the title or heads unless they are unavoidable.
			
						
			\subsection{Optimizing \ref{FCFS} with Mandelbrot  Symmetry Property}\label{SYM}
			Define abbreviations and acronyms the first time they are used in the text, 
			even after they have been defined in the abstract. Abbreviations such as 
			IEEE, SI, MKS, CGS, ac, dc, and rms do not have to be defined. Do not use 
			abbreviations in the title or heads unless they are unavoidable.
						
	\subsection{Implementation of Partition Schemes}
	Number equations consecutively. To make your 
	equations more compact, you may use the solidus (~/~), the exp function, or 
	appropriate exponents. Italicize Roman symbols for quantities and variables, 
	but not Greek symbols. Use a long dash rather than a hyphen for a minus 
	sign. Punctuate equations with commas or periods when they are part of a 
	sentence, as in:
	\begin{equation}
	a+b=\gamma\label{eq}
	\end{equation}
	
	Be sure that the 
	symbols in your equation have been defined before or immediately following 
	the equation. Use ``\eqref{eq}'', not ``Eq.~\eqref{eq}'' or ``equation \eqref{eq}'', except at 
	the beginning of a sentence: ``Equation \eqref{eq} is . . .''
	
	\subsection{Results and Discussions}
	
	
	\subsection{Conclusion}
	{\LaTeX} does not have precognitive abilities. If you put a
	\verb|\label| command before the command that updates the counter it's
	supposed to be using, the label will pick up the last counter to be
	cross referenced instead. In particular, a \verb|\label| command
	should not go before the caption of a figure or a table.
	
	Do not use \verb|\nonumber| inside the \verb|{array}| environment. It
	will not stop equation numbers inside \verb|{array}| (there won't be
	any anyway) and it might stop a wanted equation number in the
	surrounding equation. 
	
	\bibliographystyle{ieeetr}
	\bibliography{sample}




\end{document}
